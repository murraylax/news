\documentclass{beamer}
\usepackage{beamerthemeshadow}

%\documentclass{article}
%\usepackage{beamerarticle}
%\usepackage{graphicx}

\usepackage{verbatim}
%\usepackage{lastpage}
\usepackage{xcolor}
\usepackage{pgf}
\usepackage{colortbl}

\newcommand{\bi}{\begin{itemize}}
\newcommand{\ei}{\end{itemize}}
\newcommand{\be}{\begin{enumerate}}
\newcommand{\ee}{\end{enumerate}}
\newcommand{\bd}{\begin{description}}
\newcommand{\ed}{\end{description}}
\newcommand{\prbf}[1]{\textbf{#1}}
\newcommand{\prit}[1]{\textit{#1}}
\newcommand{\beq}{\begin{equation}}
\newcommand{\eeq}{\end{equation}}
\newcommand{\bdm}{\begin{displaymath}}
\newcommand{\edm}{\end{displaymath}}

\newcommand{\ft}[1]{
  \frametitle{\begin{tabular}{p{4.2in}r} \textcolor{white}{#1} & \small{\insertframenumber / \inserttotalframenumber} \end{tabular}}
  %\frametitle{#1}
  \setbeamercovered{transparent=18}
}

\newcommand{\stepinv}{\setbeamercovered{invisible}}
\newcommand{\stopinv}{\setbeamercovered{transparent=18}}
\newcommand{\uncoverinv}[1]
{
  \setbeamercovered{invisible}
  \uncover<+->{#1}
  \setbeamercovered{transparent=18}
}
\newcommand{\ans}[1]{\textcolor{blue}{#1}}
\newcommand{\ansinv}[1]
{
  \setbeamercovered{invisible}
  \uncover<+->{\textcolor{blue}{#1}}
  \setbeamercovered{transparent=18}
}
\newcommand{\setinv}{\setbeamercovered{invisible}}
\newcommand{\setvis}{\setbeamercovered{transparent=18}}
\newcommand{\centerpic}[2]
{
  \begin{center}
  \includegraphics[#1]{#2}
  \end{center}
}
\newcommand{\h}[1]{\hat{#1}}
\newcommand{\ds}{\displaystyle}

%\definecolor{light}{rgb}{0.17,0.55,0.35}
\newcommand{\hl}[1]{\alt<#1>{\rowcolor{light}\hspace*{-2.1pt}} {\hspace*{-2.1pt}} }

\definecolor{mycolor}{rgb}{0.17,0.55,0.35}
\usecolortheme[named=mycolor]{structure}

\title[Learning and Judgment Shocks in U.S. Business Cycles]{Learning and Judgment Shocks\\ in U.S. Business Cycles}
\author[James Murray, University of Wisconsin - La Crosse]{James Murray\\Department of Economics\\University of Wisconsin - La Crosse}
\date{Tuesday, March 15, 2011}

\begin{document}

\frame{\titlepage}
%\maketitle
\setcounter{framenumber}{0}

\section{Introduction}
\subsection{Purpose}

\frame
{
  \ft{Purpose}

  \begin{small}
  \uncover<+->{
  \begin{block}{Explain Expectations}
  \bi
  \item \textbf{Learning}: type of adaptive expectations, agents collect past data and run regressions.
  \item \textbf{Judgment}: agents adjust their expectations based on...
    \bi
    \item something in the news (war in Libya, earthquake in Japan),
    \item outcome of an election,
    \item complete nonsense.
    \ei
  \ei
  \end{block}
  }

  \uncover<+->{
  \begin{block}{Explain Macroeconomic Fluctuations}
    \be
    \item How is macroeconomic volatility in U.S. is explained by typical structural shocks versus judgment shocks.
    \item How much of judgment is explained by actual events versus judgment shocks.
    \ee
  \end{block}
  \end{small}
  }
}

\subsection{Expectations Framework}
\frame
{
  \ft{Expectations Framework}
  \uncover<+->{
  \begin{block}{Constant Gain Learning}
    \bi
    \item Agents' expectations are informed by least-squares forecasts based on past data.
    \item Forecasts can be directly mapped to past data on observable variables: output gap, inflation, interest rates.
    \ei
  \end{block}
  }

  \uncover<+->{
  \begin{block}{Expectation = Forecast + Judgment}.
    \bi
    \item Judgment may be informative, include relevant information not in past data.
    \item Judgment may be ill-informed (destabilizing, independent stochastic shock)
    \item Agents' actual expectations are mapped to data from Survey of Professional Forecasters.
    \ei
  \end{block}
  }
}

\subsection{Literature}
\frame[shrink]
{
  \ft{Literature: Learning}
  \uncover<+->{
  \begin{block}{Monetary Policy}
    \bi
    \item Oraphanides and Williams (JEDC, 2005): Monetary authority was optimizing, but misinformed.
    \item Primiceri (QJE, 2006): Monetary authority misinformed, expectations improved with time.
    \ei
  \end{block}
  }

  \uncover<+->{
  \begin{block}{Explaining Volatility}
    \bi
    \item Milani (2008): Time varying expectations.
    \item Bullard and Singh (2007): bad luck + Bayesian learning.
    \ei
  \end{block}
  }

  \uncover<+->{
  \begin{block}{Estimation}
    \bi
    \item Milani (JME, 2007): Explains persistence.
    \item Slobodyan and Wouters (2009): DSGE models with learning can fit data better than RE.
    \ei
  \end{block}
  }
}

\frame
{
  \ft{Literature: Judgment}
  \uncover<+->{
  \begin{block}{Central Banking Policy}
    \bi
    \item Reifschneider, Stockton, and Wilcox (1997)
    \item Svensson (2005)
    \ei
  \end{block}
  }

  \uncover<+->{
  \begin{block}{Exuberance Equilibria}
    \bi
    \item Bullard, Evans, Honkapohja (2008), (2010).
    \item Judgment is independent from fundamentals: purely destabilizing.
    \ei
  \end{block}
  }
   
  \uncover<+->{ 
  \begin{block}{Empirical Evaluation}
    \bi
    \item Missing?
    \ei
  \end{block}
  }
  
}

\section{Framework}
\subsection{New Keynesian Model}

\frame
{
  \ft{Optimal Consumer Behavior}
  \begin{block}{Utility maximization conditions}
  \uncover<+->{
    (Special case) Euler equation: $u'(c_t) = \beta E_t u'(c_{t+1}) \frac{(1+r_t)}{(1+\pi_{t+1})}$\\ 
    [0.5pc] \hline 
    ~ \\[0.5pc]
  }

  \uncover<+->{
    (Linearized) extended model:
    \vspace*{-0.5pc}
    \bdm \begin{array}{c} 
      \tilde{\lambda}_{t} = E_t \tilde{\lambda}_{t+1} + \h{r}_t - E_t \pi_{t+1} - r_t^n, \\ [0.8pc]
      \tilde{\lambda}_t = \frac{1}{ (1-\beta \eta)(1-\eta)}\left[ \beta \eta  E_t \tilde{y}_{t+1} - (1+\beta \eta^2) \tilde{y}_t + \eta \tilde{y}_{t-1} \right] 
    \end{array} \edm
    \vspace*{-0.5pc}
  }
  \end{block}

  \uncover<+->{
  \begin{block}{Notation}
  \begin{columns}

  \column[T]{2in}
  $\tilde{\lambda}_{t}$: marginal utility of income.\\
  $\tilde{y}_t$: output gap.\\
  $\h{r}_t$: nominal interest rate.\\
  $\pi_{t}$: inflation.\\
  
  \column[T]{2in}
    $\eta \in [0,1)$: habit.\\
    $\beta \in (0,1)$: discount rate.\\
    $r_t^n$: natural rate shock \\
  
  \end{columns}
  \end{block}
  }
}

\frame
{
  \ft{Producer Behavior}
  \uncover<+->{
  \begin{block}{Profit Maximizing Condition}
    \bi
    \item Firms choose prices (firms have market power)
    \item Firms only infrequently update prices.
    \item Consider expectations of future inflation.
    \item Aggregate supply depends on price level.
    \ei
    \bdm \pi_t = \frac{1}{1+\beta \gamma} \left[ \gamma \pi_{t-1} + \beta E_t \pi_{t+1} + \kappa (\tilde{y}_t - \mu \tilde{\lambda}_t) + u_t\right] \edm
  \end{block}
  }

  \uncover<+->{
    \begin{block}{Notation}
      \bi
      \item Cost push shock: $u_t$.
      \item $\gamma \in [0,1)$: price indexation. 
      \item $\kappa \in (0,\infty)$: price flexibility.
      \ei
    \end{block}
  }
}

\frame
{
  \ft{Monetary policy}
  \uncover<+->{
  \begin{block}{Taylor (1993) Rule}
    \bi
    \item Fed raises interest rates when output above potential.
    \item Fed raises interest rates when inflation above target.
    \item Fed gradually adjusts interest rate.
    \ei

  \bdm \h{r}_t = \rho_r \h{r}_{t-1} + (1-\rho_r) \left(\psi_{\pi} E_t \pi_{t+1} + \psi_y E_t \tilde{y}_{t+1} \right) + \epsilon_{r,t} \edm
  \end{block}
  }

  \uncover<+->{
  \begin{block}{Notation}
    \bi
    \item $\epsilon_{r,t}$: monetary policy shock.
    \item $\psi_{\pi} \in (0,\infty)$: feedback on inflation.
    \item $\psi_{y} \in (0,\infty)$: feedback on output.
    \item $\rho_r \in (0,1)$: gradual adjustment.
    \ei
  \end{block}
  }
}

\frame
{
  \ft{Linear Model}
  \bi
  \item Log-linearized New Keynesian model has the structural form:
  \bdm \begin{array}{c} 
    \Omega_{0} x_t = \Omega_{1} x_{t-1} + \Omega_{2} x_{t+1}^e + \Omega_{3} x_{t+2}^e + \Psi z_t \\ \\
    z_t = A z_{t-1} + \epsilon_t 
    \end{array} \edm

  \item All observable by the agents: $x_t = [\tilde{y}_t~ \pi_t~ \h{r}_t]'$
  \item Shocks not observable to agents that learn: $z_t = [r_t^n~ u_t~ \epsilon_{r,t}]'$
  \item Rational expectations solution:
  \bdm E_t x_{t+1} = G x_{t} + H z_t \edm
  \item Learning: agents estimate $G$ with by running a regression.
  \ei
}

\subsection{Expectations}
\frame
{
  \ft{Learning Regressions}
  \uncover<+->{
  \begin{block}{Regression Notation}
  \bi
  \item Let $Y_{\tau} \in \{\tilde{y}_t,~ \pi_{\tau}~ \hat{r}_{\tau}\}$ denote one of the dependent variables agents want to forecast.
  \item Let $X_{\tau} = [1~ \tilde{y}_{\tau-1}~ \pi_{\tau-1}~ \hat{r}_{\tau-1}]'$ denote vector of explanatory variables.
  \item Let $\hat{\beta}_t^{Y}$ be the row in $G$ for variable $Y_t$.
  \ei
  \end{block}
  }

  \uncover<+->{
  \begin{block}{OLS Regression}
    \bdm \hat{\beta}_t^{Y} = \left(\sum_{\tau=0}^{t-1} X_{\tau} X_{\tau}' \right)^{-1} \left(\sum_{\tau=0}^{t-1} X_{\tau}' Y_{\tau} \right) \edm
    \bdm \mbox{Econometric Forecast:~} E_t^* Y_t = X_t' \hat{\beta}_t \edm
  \end{block}
  }
}


\frame
{
  \ft{Learning Algorithm}
  \begin{small}
  \uncover<+->{
  \begin{block}{Recursive Formulation}
  The least squares regression coefficients can be rewritten as:
    \bdm \begin{array}{c}
      \hat{\beta}_t^Y = \beta_{t-1}^Y + g_t R_t^{-1} X_t' (Y_t - X_t \hat{\beta}_t) \\ [1pc] 
      R_t = R_{t-1} + g_t (X_t X_t' - R_{t-1}),
    \end{array}\edm
  where $g_t = 1/t$ is the \textbf{learning gain}.
  \end{block}
  }

  \uncover<+->{
  \begin{block}{Learning Gain}
    \bi
    \item $g_t \rightarrow 0$ as $t \rightarrow \infty$, learning disappears over time.
    \item Constant gain learning:  $g_t = g$.
    \item Learning can \textit{always} lead to changes in expectations.
    \item Allows agents to learn about structural changes.
    \ei
  \end{block}
  }
  \end{small}
}

\frame
{
  \ft{Expectations}
  \uncover<+->{
  \begin{block}{Data Requirements}
  \bi
  \item Recall rational expectations: $E_t x_{t+1} = G x_{t} + H z_t$
  \item Learning agents have data on $x_t$, cannot ``get data'' on structural shocks, $z_t$.
  \ei
  \end{block}
  }
  
  \uncover<+->{
  \begin{block}{Expectations: Learning with Judgment}
  \bi
  \item Judgment may include evidence of structural shocks that are evident from news or current events.
  \item Expectations: sum of econometric forecasts ($E_t^*x_{t+1}$) and judgment ($\eta_t$).
  \ei
  \bdm x_{t+1}^e = E_t^*x_{t+1} + \eta_t \edm
  \end{block}
  }
}


\frame
{
  \ft{Judgment}
  \uncover<+->{
    \begin{block}{Evolution of Judgment}
      Judgment, $\eta_t$, is possibly informed by current structural shocks, and subject to is own shock:
    \bdm \begin{array}{c} \ds \eta_t = \Phi z_t + \zeta_t, \\ [0.7pc]
      \ds \zeta_{y,t} = \rho_{\zeta,y} \zeta_{y,t-1} + \xi_{y,t}, \\ [0.7pc]
      \ds \zeta_{\pi,t} = \rho_{\zeta,\pi} \zeta_{\pi,t-1} + \xi_{\pi,t},
    \end{array} \edm
    \end{block}
  }

  \uncover<+->{
    \begin{block}{Notation}
    \bi
    \item $\eta_t$ is 2x1 vector, includes judgment on $\tilde{y}_{t+1}^e$ and  $\pi_{t+1}^e$.
    \item $\Phi$: dependence of judgment on actual structural shocks.
    \item $\zeta_t$: judgment shocks.
    \ei
    \end{block}
  }

}

\section{Estimation}
\subsection{Model Estimation}
\frame
{
  \ft{Estimation}
  \bi
  \item Bayesian Estimation - Metropolis Hastings Simulation Procedure.
  \item Quarterly data from 1968:Q3 through 2007:Q1 on 
    \bi
    \item Output gap: measured by Congressional Budget Office.
    \item GDP deflator inflation rate.
    \item Federal funds rate.
    \item Survey of Professional Forecasters One-Quarter ahead forecast on real GDP.
    \item Survey of Professional Forecasters One-Quarter ahead forecast on GDP deflator.
    \ei
  \item Pre-sample (1954:Q3 - 1968:Q2) data on first three variables initialize VAR(1) learning forecasts.
  \ei
}

\frame[shrink]
{
\ft{Parameter Estimates}
\begin{columns}
\column[T]{2.5in}
\begin{block}{New Keynesian Model Parameters}
\begin{footnotesize}
\begin{table}
\begin{center}
\begin{tabular}{l|c|cc}
 & Median & 5th PCT & 95th PCT \\ \hline 
\only<2>{~\rowcolor{yellow}} $\eta$ & 0.0715 & 0.0207 & 0.1420 \\
~$\sigma$ & 2.9178 & 2.2683 & 3.5847 \\ 
~$\mu$ & 2.0691 & 1.3988 & 2.8363 \\ 
~$\kappa$ & 0.0278 & 0.0161 & 0.0432 \\ 
\only<3>{~\rowcolor{yellow}} $\gamma$ & 0.8465 & 0.7241 & 0.9146 \\ 
~$\rho_r$ & 0.9210 & 0.8578 & 0.9572 \\ 
~$\psi_y$ & 0.3185 & 0.1054 & 0.5845 \\ 
~$\psi_{\pi}$ & 1.5262 & 1.2789 & 1.7665 \\ 
\only<4>{~\rowcolor{yellow}} $\rho_n$ & 0.9798 & 0.9629 & 0.9925 \\ 
\only<5>{~\rowcolor{yellow}} $\rho_u$ & 0.0619 & 0.0146 & 0.2714 \\ 
~$\sigma_{n}$ & 0.0302 & 0.0236 & 0.0376 \\ 
~$\sigma_{u}$ & 0.0039 & 0.0035 & 0.0045 \\ 
~$\sigma_{r}$ & 0.0037 & 0.0033 & 0.0040 \\  \hline
\end{tabular}
\end{center}
\end{table}
\end{footnotesize}
\end{block}

\uncover<+->{
\column[T]{1.5in}
\begin{block}{Comments}
\be
\item<2> Low persistence due to habit formation.
\item<3> High inflation persistence.
\item<4> High persistence in natural rate shock.
\item<5> Low persistence in cost-push shock.
\ee
\end{block}
}
\end{columns}

}

\frame
{
\ft{Parameter Estimates}
\begin{columns}
\column[T]{2.5in}
\begin{block}{Expectation Parameters}
\begin{footnotesize}
\begin{table}
\begin{center}
\begin{tabular}{l|c|cc}
 & Median & 5th PCT & 95th PCT \\ \hline 
\only<2>{~\rowcolor{yellow}} $g$ & 0.0232 & 0.0103 & 0.0439 \\ 
\only<3>{~\rowcolor{yellow}} $\rho_{\zeta,y}$ & 0.7322 & 0.4884 & 0.9385 \\ 
\only<3>{~\rowcolor{yellow}} $\rho_{\zeta,\pi}$ & 0.8729 & 0.7896 & 0.9460 \\ 
~$\sigma_{\zeta,y}$ & 0.0090 & 0.0082 & 0.0100 \\ 
~$\sigma_{\zeta,\pi}$ & 0.0050 & 0.0045 & 0.0055 \\ 
\only<4>{~\rowcolor{yellow}} $\phi_{y,n}$ & -0.2220 & -0.2937 & -0.1466 \\ 
\only<5>{~\rowcolor{yellow}} $\phi_{y,u}$ & 0.0916 & -0.2233 & 0.3346 \\ 
\only<5>{~\rowcolor{yellow}} $\phi_{y,r}$ & -0.0394 & -0.2990 & 0.3760 \\ 
\only<4>{~\rowcolor{yellow}} $\phi_{\pi,n}$ & 0.0252 & 0.0015 & 0.0503 \\ 
\only<4>{~\rowcolor{yellow}} $\phi_{\pi,u}$ & -0.2890 & -0.4411 & -0.1428 \\ 
\only<5>{~\rowcolor{yellow}} $\phi_{\pi,r}$ & -0.0679 & -0.2102 & 0.0934 \\ \hline
\end{tabular}
\end{center}
\end{table}
\end{footnotesize}
\end{block}

\column[T]{1.5in}
\uncover<+->{
\begin{block}{Comments}
\be
\item<2> Typical learning gain $\sim 43 obs. \sim 11 years$.
\item<3> High judgment persistence.
\item<4> Informed judgment (non-zero).
\item<5> Judgment not informed.
\ee
\end{block}
}
\end{columns}
}

\subsection{Judgment}
\frame{
  \ft{Informative Content in Judgment}
  \uncover<+->{
  \begin{block}{Judgment}
    Recall, judgment is a linear combination of concurrent structural shocks and its own stochastic disturbance:
    \bdm \begin{array}{lc} \ds \mbox{Judgment:~} & \eta_t = \Phi z_t + \zeta_t, \\ 
      \ds \mbox{Disturbance:~} & \zeta_t = \Rho \zeta_{t-1} + \xi_t,
    \end{array} \edm
  \end{block}
  }

  \uncover<+->{
  \begin{block}{Variance Decomposition}
    What percentage of the variability in judgment ($\eta_t$) is,
      \be
      \item informed by concurrent structural shocks ($z_t$)?
      \item stochastic disturbances ($\xi_t$)?
      \ee
    Uses the estimates parameters in $\Phi$, $\rho_{\zeta,y}$, $\rho_{\zeta,\pi}$ and the variances of $z_t$, $\xi_{y,t}$, $\xi_{\pi,t}$.
  \end{block}
  }
}


\frame[shrink]
{
  \ft{Informative vs. Stochastic Judgment}

\begin{block}{Variance Decomposition}
\begin{footnotesize}
\begin{center}
\begin{tabular}{l|c|cc} 
~Stochastic Shock & Output Judg. & Inflation Judg. \\ \hline
\only<2>{~\rowcolor{yellow}} Natural Rate Shock & 86.5 \% & 12.1\%  \\
\only<3>{~\rowcolor{yellow}} Cost-Push Shock & 0.0\% & 1.1\%  \\
~Monetary Policy Shock  & 0.0\% & 0.0\% \\
\only<4>{~\rowcolor{yellow}} Output Judgment Shock  & 13.5\% & -- \\
\only<5>{~\rowcolor{yellow}} Inflation Judgment Shock  & -- & 86.7\% \\ \hline
~Total & 100.00\% & 100.00\% \\ \hline
\end{tabular}
\end{center}
\end{footnotesize}
\end{block}

\uncover<+->{
\begin{block}{Comments}
\small{
  \be
  \item<2> Expectations (judgment) are informed by the natural rate shock.
  \item<3> Expectations are not informed by cost-push shock.
  \item<4> Some variability in judgment for output are from stochastic disturbances.
  \item<5> Most of the variability in judgment for inflation are from stochastic disturbances.
  \ee
}
\end{block}
}
}

\subsection{Impulse Responses}
\frame
{
  \ft{Impulse Responses: Output Judgment Shock}
  \begin{block}{Response to Output Gap from Output Judgment Shock}
  \begin{center}
    \begin{tabular}{cc}
    \includegraphics[width=1.9in,height=1.45in]{images/Irf16_Output_Gap_Output_Judgment_Shock.png} &
    \includegraphics[width=1.9in,height=1.45in]{images/RMS16_Output_Gap_Output_Judgment_Shock.png} \\
    \end{tabular}
  \end{center}
  \end{block}

  \begin{block}{Comments}
    \small{
    \bi
    \item Output judgment shock increases output.
    \item Larger IRF's coincide with 1980s volatility, rapid growth of 1990s, slow growth in 2000s, slow recovery 2010 recession.
    \ei
    }
  \end{block}
}


\frame
{
  \ft{Impulse Responses: Output Judgment Shock}
  \begin{block}{Response to Inflation from Output Judgment Shock}
  \begin{center}
    \begin{tabular}{cc}
    \includegraphics[width=1.9in,height=1.45in]{images/Irf16_Inflation_Output_Judgment_Shock.png} &
    \includegraphics[width=1.9in,height=1.45in]{images/RMS16_Inflation_Output_Judgment_Shock.png} \\
    \end{tabular}
  \end{center}
  \end{block}

  \begin{block}{Comments}
    \small{
    \bi
    \item Output judgment shock increases inflation.
    \item Larger IRF's occur during same time periods.
    \ei
    }
  \end{block}
}

\frame
{
  \ft{Impulse Responses: Inflation Judgment Shock}
  \begin{block}{Response to Output Gap from Inflation Judgment Shock}
  \begin{center}
    \begin{tabular}{cc}
    \includegraphics[width=1.9in,height=1.45in]{images/Irf16_Output_Gap_Inflation_Judgment_Shock.png} &
    \includegraphics[width=1.9in,height=1.45in]{images/RMS16_Output_Gap_Inflation_Judgment_Shock.png} \\
    \end{tabular}
  \end{center}
  \end{block}

  \begin{block}{Comments}
    \small{
    \bi
    \item Inflation judgment shock increases output (reduces expected real interest rate).
    \item Inflation judgment IRFs on output have diminished over time.
    \ei
    }
  \end{block}
}


\frame
{
  \ft{Impulse Responses: Inflation Judgment Shock}
  \begin{block}{Response to Inflation from Inflation Judgment Shock}
  \begin{center}
    \begin{tabular}{cc}
    \includegraphics[width=1.9in,height=1.45in]{images/Irf16_Inflation_Inflation_Judgment_Shock.png} &
    \includegraphics[width=1.9in,height=1.45in]{images/RMS16_Inflation_Inflation_Judgment_Shock.png} \\
    \end{tabular}
  \end{center}
  \end{block}

  \begin{block}{Comments}
    \small{
    \bi
    \item Inflation judgment shock increases inflation.
    \item Response is not symmetric over time.  Largest in last few years of the sample.
    \ei
    }
  \end{block}
}

\frame
{
  \ft{Comparing Impulse Responses}
  \begin{columns}
  \column[T]{2.5in}
  \begin{block}{Average Root Mean Squared Responses\\(One Std.Dev. Shock)} 
    \begin{footnotesize}
    \begin{tabular}{l|c|c}
      \multicolumn{3}{c}{First Four Periods of IRF} \\ \hline
      ~Shock & Output & Inflation \\ \hline 
      ~Natural Rate & 0.6018 & 0.1981  \\ 
      \only<3>{~\rowcolor{yellow}} Cost-Push & 0.1697 & 1.0864  \\ 
      ~Monetary Policy & 0.6364 & 0.1787  \\ 
      \only<2,4>{~\rowcolor{yellow}} Output Judgment & 1.2952 & 0.3662  \\ 
      \only<4>{~\rowcolor{yellow}} Inflation Judgment & 0.2911 & 0.3029  \\ 
      \hline
    \end{tabular}\\

    \begin{tabular}{l|c|c}
      \multicolumn{3}{c}{First Sixteen Periods of IRF} \\ \hline
      ~Shock & Output & Inflation \\ \hline 
      ~Natural Rate & 0.9918 & 0.6533 \\ 
      \only<3>{~\rowcolor{yellow}} Cost-Push & 0.1870 & 0.6953 \\ 
      ~Monetary Policy & 0.7742 & 0.4854 \\ 
      \only<2,4>{~\rowcolor{yellow}} Output Judgment & 1.0627 & 0.6060 \\ 
      \only<4>{~\rowcolor{yellow}} Inflation Judgment & 0.3353 & 0.4694 \\ 
      \hline
    \end{tabular}
    \end{footnotesize}
  \end{block}

  \column[T]{1.5in}
  \uncover<+->{
  \begin{block}{Comments}
    \begin{small}
    \bi
    \item<2> Output judgment shock has largest average impact on output.
    \item<3> Cost-push shock has largest impact on inflation.
    \item<4> Both output judgment and inflation judgment influence inflation dynamics.
    \ei
    \end{small}
  \end{block}
  }
  \end{columns}
}

\section{}
\subsection{Conclusion}
\frame
{
  \ft{Conclusions}
  \bi
  \item Judgment is a significant source of persistence for output and inflation.
  \item Inflation judgment is mostly dependent on stochastic disturbances.
  \item Output judgment is largely informed by concurrent natural rate shock.
  \item Both output and inflation judgment shocks are important drivers of business cycle fluctuations, along with natural rate shock and cost-push shock.
  \ei
}


\end{document}

